\chapter{Introduction}
\label{chapter:introduction}


\lipsum[1-7]

\section{Clinical need}

\lipsum[1-2]\cite{HidalgoEquivitalEQ02Online,poh2011advancements,poh2010non,verkruysse2008remote,wieringa2005contactless,wiki:bayerfilter,nonin9560oem,Mas_Fernandez_2003,hartley2003multiple}

\subsection{Sub section 1}

\lipsum[3-6]\cite{schmidt1997animal,4036907,Orbanz2008,varjavand2002interactive,report:PCESC2005,HSU_05_2012,collins2015relating,abdi2010principal}

\subsubsection{Subsub section title}

\lipsum[3-5]

\subsubsection{Subsub section title}

\lipsum[8-10]

\subsection{Sub section 2}

\lipsum[10-12]\cite{cote1988effect,aravindhan2000sulfhemoglobinemia,clayton1991pulse,clayton1991comparison,webb1991potential,212885,964165,Chaichulee2017FG}

\section{Objectives}

\lipsum[2-4]

\section{Contributions}

\lipsum[2-4]

\section{Outline of document}

\Cref{chapter:literature_review} discusses .... . The clinical study undertaken as part of this project is presented in \Cref{chapter:dataset} with a detailed description of the patient population, the data collection process and an overview of the main vital signs used as the reference data....

\lipsum[2-4]
